\section{Regolamentazione delle Proposte}

% Definizione delle tipologie di proposte

\begin{defn}[Proposte]
	Proposte.
\end{defn}

\begin{defn}[Proposte Classiche]
	Proposte Classiche.
\end{defn}

\begin{defn}[Proposte Pecuniarie]
	Vengono definite \textbf{proposte pecuniarie} tutte le proposte che coinvolgono flussi di denaro:
	\begin{itemize}
		\item in uscita dalla cassa dello Student Branch per spese ad uso dei membri dello Student Branch.
		\item in uscita dalla cassa dello Student Branch per spese destinate a terzi.
		\item in uscita dai portafogli dei membri attivi dello Student Branch per spese ad uso degli stessi. % DA SISTEMARE
		\item in uscita dai portafogli dei membri attivi dello Student Branch verso la cassa dello Student Branch.
	\end{itemize}
\end{defn}

\begin{defn}[Proposte Straordinarie]
	Vengono definite \textbf{proposte straordinarie} le proposte
\end{defn}

% Differenze fra le due tipologie

% Modalità di presa delle decisioni.
\section{Regolamentazione delle Decisioni}

\paragraph{Chi prende le decisioni}

\paragraph{Modalità di votazione}