\documentclass[]{article}
\usepackage[utf8]{inputenc}
\usepackage[italian]{babel}

\usepackage{amsthm}

\theoremstyle{definition}
\newtheorem{defn}{Definizione}[section]
\newtheorem{regl}{Regola}[section]

\renewcommand{\i}[1]{\textit{#1}}
\renewcommand{\b}[1]{\textbf{#1}}

%opening
\title{Regolamento}
\author{Executive Committee: Lorenzo Nodari, Stefano Prandini, Massimo Bono, Michele Dusi}
\date{\today}

\begin{document}

\maketitle

\newpage

\section{Regolamentazione dei Membri}

\begin{defn}[Iscritta IEEE]
	Una persona viene definita \b{iscritta IEEE} se possiede un account con iscrizione all'IEEE.
\end{defn}

\begin{defn}[Esterno]
	Una persona viene definita \b{esterna} se non è una persona iscritta IEEE.
\end{defn}

\begin{defn}[Status]
	Viene definito \b{status} di una persona iscritta IEEE una qualunque etichetta associabile a tale persona.
\end{defn}

\begin{defn}[Status parametrizzato]
Dato un insieme generico $A$ viene definito \b{status parametrizzato} di una persona iscritta IEEE una coppia $(s,v)$ dove $s$ è uno status dell'iscritto IEEE, e $v$ è un elemento dell'insieme $A$.
\end{defn}

\begin{regl}
	Un iscritto IEEE acquisisce lo status di \b{\i{membro attivo}} dello Student Branch se si verificano contemporaneamente le seguenti condizioni:
	\begin{itemize}
		\item Si presenta ad altri \i{membri attivi} dello Student Branch.
		\item Manifesta ad altri \i{membri attivi} dello Student Branch la volontà di partecipare alle attività dello Student Branch.
		\item In passato non ha mai acquisito lo status di \i{membro passivo}.
	\end{itemize}
\end{regl}

\begin{regl}
	Un iscritto IEEE guadagna lo status di \b{\i{Membro Passivo}} dello Student Branch se gode dello status di \i{membro attivo} dello Student Branch e si verifica una qualunque delle seguenti condizioni:
	\begin{itemize}
		\item L'ultima presenza registrata nello Student Branch è antecedente di tre mesi.
		\item 
	\end{itemize}
	Nell'istante in cui un iscritto IEEE guadagna tale status, egli perde immediatamente lo status di \i{membro attivo} dello Student Branch.
\end{regl}

\section{Regolamentazione delle Proposte}

% Definizione delle tipologie di proposte

\begin{defn}[Proposte]
	Proposte.
\end{defn}

\begin{defn}[Proposte Classiche]
	Proposte Classiche.
\end{defn}

\begin{defn}[Proposte Pecuniarie]
	Vengono definite \textbf{proposte pecuniarie} tutte le proposte che coinvolgono flussi di denaro:
	\begin{itemize}
		\item in uscita dalla cassa dello Student Branch per spese ad uso dei membri dello Student Branch.
		\item in uscita dalla cassa dello Student Branch per spese destinate a terzi.
		\item in uscita dai portafogli dei membri attivi dello Student Branch per spese ad uso degli stessi. % DA SISTEMARE
		\item in uscita dai portafogli dei membri attivi dello Student Branch verso la cassa dello Student Branch.
	\end{itemize}
\end{defn}

\begin{defn}[Proposte Straordinarie]
	Vengono definite \textbf{proposte straordinarie} le proposte
\end{defn}

% Differenze fra le due tipologie

% Modalità di presa delle decisioni.
\section{Regolamentazione delle Decisioni}

\paragraph{Chi prende le decisioni}

\paragraph{Modalità di votazione}

\section{Regolamentazione delle Sanzioni}

Struttura (da valutare):
\begin{enumerate}
	\item Elenco delle "colpe"
	\item Elenco delle relative "sanzioni"
\end{enumerate}


\end{document}
