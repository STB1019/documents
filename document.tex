\documentclass[pdf]{article}
\usepackage[utf8]{inputenc}
\usepackage[english]{babel}

\usepackage{amsthm}
\usepackage{xparse}
\usepackage{etoolbox}
\usepackage{xcolor}

\theoremstyle{definition}
\newtheorem{defn}{Definizione}[section]
\newtheorem{regl}{Regola}[section]

\NewDocumentCommand{\todo}{m}{
	\footnote{\textcolor{red}{#1}}%
}

\NewDocumentCommand{\EC}{}{
	Executive Commitee%
}

\renewcommand{\i}[1]{\textit{#1}}
\renewcommand{\b}[1]{\textbf{#1}}

%opening
\title{Regolamento}
\author{Executive Committee: Lorenzo Nodari, Stefano Prandini, Massimo Bono, Michele Dusi}
\date{\today}

\begin{document}

\maketitle

\newpage

\section{Regolamentazione dei Membri}

\begin{defn}[Iscritta IEEE]
	Una persona viene definita \b{iscritta IEEE} se possiede un account con iscrizione all'IEEE.
\end{defn}

\begin{defn}[Esterno]
	Una persona viene definita \b{esterna} se non è una persona iscritta IEEE.
\end{defn}

\begin{defn}[Status]
	Viene definito \b{status} di una persona iscritta IEEE una qualunque etichetta associabile a tale persona.
\end{defn}

\begin{defn}[Status parametrizzato]
Dato un insieme generico $A$ viene definito \b{status parametrizzato} di una persona iscritta IEEE una coppia $(s,v)$ dove $s$ è uno status dell'iscritto IEEE, e $v$ è un elemento dell'insieme $A$.
\end{defn}

\begin{regl}
	Un iscritto IEEE acquisisce lo status di \b{\i{membro attivo}} dello Student Branch se si verificano contemporaneamente le seguenti condizioni:
	\begin{itemize}
		\item Si presenta ad altri \i{membri attivi} dello Student Branch.
		\item Manifesta ad altri \i{membri attivi} dello Student Branch la volontà di partecipare alle attività dello Student Branch.
		\item In passato non ha mai acquisito lo status di \i{membro passivo}.
	\end{itemize}
\end{regl}

\begin{regl}
	Un iscritto IEEE guadagna lo status di \b{\i{Membro Passivo}} dello Student Branch se gode dello status di \i{membro attivo} dello Student Branch e si verifica una qualunque delle seguenti condizioni:
	\begin{itemize}
		\item L'ultima presenza registrata nello Student Branch è antecedente di tre mesi.
		\item 
	\end{itemize}
	Nell'istante in cui un iscritto IEEE guadagna tale status, egli perde immediatamente lo status di \i{membro attivo} dello Student Branch.
\end{regl}

\section{Regolamentazione delle Proposte}

% Definizione delle tipologie di proposte

\begin{defn}[Proposte]
	Proposte.
\end{defn}

\begin{defn}[Proposte Classiche]
	Proposte Classiche.
\end{defn}

\begin{defn}[Proposte Pecuniarie]
	Vengono definite \textbf{proposte pecuniarie} tutte le proposte che coinvolgono flussi di denaro:
	\begin{itemize}
		\item in uscita dalla cassa dello Student Branch per spese ad uso dei membri dello Student Branch.
		\item in uscita dalla cassa dello Student Branch per spese destinate a terzi.
		\item in uscita dai portafogli dei membri attivi dello Student Branch per spese ad uso degli stessi. % DA SISTEMARE
		\item in uscita dai portafogli dei membri attivi dello Student Branch verso la cassa dello Student Branch.
	\end{itemize}
\end{defn}

\begin{defn}[Proposte Straordinarie]
	Vengono definite \textbf{proposte straordinarie} le proposte
\end{defn}

% Differenze fra le due tipologie

% Modalità di presa delle decisioni.
\section{Regolamentazione delle Decisioni}

\paragraph{Chi prende le decisioni}

\paragraph{Modalità di votazione}

\section{Regolamentazione delle Sanzioni}

Struttura (da valutare):
\begin{enumerate}
	\item Elenco delle "colpe"
	\item Elenco delle relative "sanzioni"
\end{enumerate}

\section{Cose veloci}

\subsection{iscrizione allo student branch}

\begin{enumerate}
	\item lo studente (d'ora in poi denominato con $p$) si iscrive alla IEEE;
	\item verificare che $p$ sia effettivamente iscritta all’IEEE;
	\item il segretario deve richiedere a $p$ la mail relativa al suo account IEEE ed un numero telefonico contattabile;
	\item il segretario iscrive il nuovo studente al google drive, alla mailing list, al sito, open project;
	\item $p$ deve decidere di che comitato far parte;
	\item da quel momento in poi $p$ è un membro attivo dello student branch;
\end{enumerate}

\subsection{Proposte e decisioni}

Si possono fare 3 proposte: classiche, pecuniarie e straordinarie.

le pecuniarie sono quelle proposte che comportano una spesa per ciascun membro attivo dello STB.

le classiche sono quelle proposte che non sono pecuniarie.

Le straordinarie sono proposte classiche che però devono essere decise velocemente (non possono rispettare i tempi definiti dalla procedura per la gestione delle propsote classiche).

\subsubsection{Proposte classiche}

il tutto inizia con una mail inviata al chair da parte di un qualunque membro attivo del STB. Nella mail deve essere contenuta la proposta da decidere: più la proposta è ben scritta maggiori saranno le possibilità che venga accettata. 

La mail viene inviata al chair che a prescindere la inoltra a tutti gli altri membri dell'\EC{}. I membri dell'\EC{} hanno 3 giorni di tempo per visionare la proposta ed esprimere, tramite voto, se proporre la suddetta al resto del branch o meno.
La votazione viene eseguita come segue:
è a maggioranza. Se un membro dell'\EC{} non esprime preferenza non conta. Quindi la proposta per essere accettata deve ottenere:
3/4
2/3
1/2
1/1

Nel caso in cui solo 2 membri dell'\EC{} abbiamo effettuato il voto ed essi siano in disaccordo (caso 1/2) di default la proposta viene rifiutata.

In caso di proposta rifiutata il chair deve inviare una mail al proponente in cui dice:
\begin{itemize}
	\item propsta rifiutata;
	\item motivi (proposta troppo vaga oppure inutile);
\end{itemize}

In caso di proposta accettata il chain deve inviare una mail al proponente in cui dice:
\begin{itemize}
	\item proposta accettata;
\end{itemize}

Il vice deve fare il form da proporre a tutti i membri attivi del branch, nel quale le uniche possibilità di risposta saranno SI'/NO. Ci deve essere una descrizione conforme alla proposta inviata dal proponente. 
Il vice deve fare una mail in cui avverte i membri attivi del branch che è attiva una votazione.

I membri attivi hanno una settimana di tempo per votare. Alla fine della stessa accade che:

\begin{itemize}
	\item la proposta viene accettata dal branch se e solo se raggiunge il 51\% dei votanti ed il 50\%\todo{decidere la percentuale meglio!} dei membri attivi del branch ha votato.
	\item la proposta viene rifiutata altrimenti;
\end{itemize}

In entrambi i casi viene inviata una mail di conferma per  avvisare l'esito della votazione a tutti i membri attivi del branch. La mail è inviata dal vice.

\subsubsection{Proposta pecuniaria}

il tutto inizia con una mail inviata al chair da parte di un qualunque membro attivo del STB. Nella mail deve essere contenuta la proposta da decidere: più la proposta è ben scritta maggiori saranno le possibilità che venga accettata. 

La mail viene inviata al chair che a prescindere la inoltra a tutti gli altri membri dell'\EC{}. I membri dell'\EC{} hanno 3 giorni di tempo per visionare la proposta ed esprimere, tramite voto, se proporre la suddetta al resto del branch o meno.
La votazione viene eseguita come segue:
è a maggioranza. Se un membro dell'\EC{} non esprime preferenza non conta. Quindi la proposta per essere accettata deve ottenere:
3/4
2/3
1/2
1/1

Nel caso in cui solo 2 membri dell'\EC{} abbiamo effettuato il voto ed essi siano in disaccordo (caso 1/2) di default la proposta viene rifiutata.

In caso di proposta rifiutata il chair deve inviare una mail al proponente in cui dice:
\begin{itemize}
	\item propsta rifiutata;
	\item motivi (proposta troppo vaga oppure inutile);
\end{itemize}

In caso di proposta accettata il chain deve inviare una mail al proponente in cui dice:
\begin{itemize}
	\item proposta accettata;
\end{itemize}

Il vice deve fare il form da proporre a tutti i membri attivi del branch, nel quale le uniche possibilità di risposta saranno SI'/NO. Ci deve essere una descrizione conforme alla proposta inviata dal proponente. 
Il vice deve fare una mail in cui avverte i membri attivi del branch che è attiva una votazione. Nella mail il tesoriere deve essere messo in CC.

I membri attivi hanno una settimana di tempo per votare. Alla fine della stessa accade che:

\begin{itemize}
	\item la proposta viene accettata dal branch se e solo se raggiunge il 65\% dei votanti ed il 50\%\todo{decidere la percentuale meglio!} dei membri attivi del branch ha votato.
	\item la proposta viene rifiutata altrimenti;
\end{itemize}

In entrambi i casi viene inviata una mail di conferma per  avvisare l'esito della votazione a tutti i membri attivi del branch. La mail è inviata dal vice.

Il tesoriere ha 2 settimane di tempo per ritirare i soldi dai membri attivi del branch.

\subsubsection{Proposta straordinaria}

Le proposte straordinaria è una proposta classica che però non può ottemperare ai tempi richiesti da una proposta classica.

il tutto inizia con un evento straordinario che richiede una decisione tempestiva.

Prima di tutto l'\EC{} si riunisce ed effettua una votazione peer decidere la proposta.
La votazione è eseguita come segue:
3/4
2/3
1/2 (di defeault no);
1/1

Nel caso in cui solo 2 membri dell'\EC{} abbiamo effettuato il voto ed essi siano in disaccordo (caso 1/2) di default la proposta viene rifiutata.

L'\EC{} deve inviare una mail a tutti i membri attivi del branch in cui spiega la situazione straordinaria e indice la votazione tramite form: 
Il vice deve fare una mail in cui avverte i membri attivi del branch che è attiva una votazione. Ci deve essere una descrizione conforme alla proposta inviata dal proponente. 
L'\EC{} fa un form comunque che se viene compilato da più del 50\% dei membri attivi e ha viene preso in considerazione altrimenti la decisione presa dall'\EC{} viene presa come valida.

I membri attivi hanno il tempo richiesto dalla situazione per votare (esempio: il rettore entra all'incontro del branch e dice: "tra 2 ore dovete decidere". Il tempo di votazione qua sarà di 2 ore).

Alla fine del tempo limite:

\begin{itemize}
	\item la proposta viene accettata dal branch se e solo se raggiunge il 51\% dei votanti ed il 50\%\todo{decidere la percentuale meglio!} dei membri attivi del branch ha votato pro;
	\item la proposta viene rifiutata dal branch se e raggiunge il 51\% dei votanti ed il 50\% dei mambri artivi ha votato contro;
	\item la proposta viene decisa solo dal \EC{} altrimenti;
\end{itemize}

In tutti i casi viene inviata una mail di conferma per  avvisare l'esito della votazione a tutti i membri attivi del branch. La mail è inviata dal vice.

\subsection{sanzioni}

è sanzionabile:

\begin{enumerate}
	\item mancata partecipazione all'attività del branch (un membro attivo non si fa sentire da mesi);
	\item comportamento non consono (un membro attivo rompe una sedia/ruba un computer/stalkera);
	\item non paga (un membro attivo va contro la decisione presa dal branch di comprare qualcosa e non da i soldi);
	\item non mantiene il lavoro promesso (un membro attivo non collabora nel progetto che lui stesso ha deciso di fare);
	\item non svolge i compiti richiesti dal branch (un membro di un comitato, anche execute, non espleta correttamente i suoi compiti);
\end{enumerate}

In seguito segue la procedura standard da eseguire per ciascuna delle azioni.

\subsubsection{Mancata partecipazione}

\todo{cosa vuol dire mancata partecipazione?}

In caso un membro attivo qualunque dello STB non partecipi per più di 3 mesi, il membro sarà degradata a membro passivo. Inoltre il \todo{chi invia la mail?} dovrà inviare una mail notificando l'utente di tale degradamento.

Il soggetto può \textbf{explicitamente} richiedere di essere reintrodotto tra i membri attivi dello STB tramite mail.

\todo{forse qui è meglio usare una vita?}
Se, nonostante ciò, egli non si rende attivo per altri 2 mesi nello STB, tale membro sarà ridegradato a membro passivo.

Anche in questo secondo caso il membro può richiedere di essere reintrodotto tra i membri attivi dello STB tramite mail.

Se, nonostante ciò, egli non si rende attivo per un altro mese, tale membro sarà espulso in via definitiva dallo STB.

\subsubsection{Comportamento non consono}

\todo{cosa vuol dire comportamento non consono?}

Esistono 2 casistiche per la gestione del comportamento non consono:

\begin{itemize}
	\item comportamento non consono classico;
	\item comportamento non consono grave: un membro attivo qualunque dello STB è autore di un atto gravissimo;
\end{itemize}

La classificazione tra un comportamento non consono e classico è a discrezione dell'\EC{}. L'\EC{} effettuerà una votazione per stabilire se un comportamento è da ritenersi grave oppure no. La votazione è a maggioranza e, in caso di assenti o non votanti nell'\EC{} si utilizzano le soglie dettate in precedenza, ossia:
3/4
2/3
1/2 (default sì)
1/1

in caso siano stati espressi solo 2 voti ed i loro voti siano contrastanti, vince il sì che sta per "sì, il comportamento è ritenuto grave".

In caso l'accusato è un membro dell'\EC{}, egli non parteciperà alla votazione.

\paragraph{comportamento non consono classico}

In caso di comportamento non consono classico si dovranno incontrare di persona l'accusato, eventuali testimoni, l'accusa e l'\EC{}.

Dopo un dibattito che può durare fino ad un'ora l'\EC{} si riunisce in via privata per discutere se incolpare o meno l'accusato. In caso l'accusato sia un membro dell'\EC{} quest'ultimo perde la possibilità di partecipare alla riunione privata.

A questo punto l'\EC{} si manifesta pubblicamente con l'accusato.
I possibili esiti possono essere:
\begin{itemize}
	\item assoluzione completa: l'accusato si è difeso bene e può tornare a lavorare come prima;
	\item assoluzione parziale: l'\EC{} pone delle condizioni per il rientro nel gruppo di lavoro e l'accusato accetta suddette condizioni (patteggiamento);
	\item rifiuti delle condizioni: l'\EC{} pone delle condizioni per il rientro nel gruppo dell'accusato ma l'accusato rifiuta tali condizioni: in questo caso l'accusato viene espulso in via definitiva;
	\item colpevolezza: l'\EC{} non pone alcuna condizioni per l'accusato ma lo condanna direttamente all'espulsione;
\end{itemize}

\paragraph{comportamento non consono grave}

In caso di comportamento non consono grave si dovranno incontrare di persona l'accusato, eventuali testimoni, l'accusa, il conselor  e l'\EC{}.

Dopo un dibattito che può durare fino ad un'ora l'\EC{} ed il conselor si riunisce in via privata per discutere se incolpare o meno l'accusato. In caso l'accusato sia un membro dell'\EC{} quest'ultimo perde la possibilità di partecipare alla riunione privata.

A questo punto l'\EC{} ed il conselor si manifesta pubblicamente con l'accusato.
I possibili esiti possono essere:
\begin{itemize}
	\item assoluzione completa: l'accusato si è difeso bene e può tornare a lavorare come prima;
	\item assoluzione parziale: l'\EC{} ed il conselor pongono delle condizioni per il rientro nel gruppo di lavoro e l'accusato accetta suddette condizioni (patteggiamento);
	\item rifiuti delle condizioni: l'\EC{} ed il conselor pongono delle condizioni per il rientro nel gruppo dell'accusato ma l'accusato rifiuta tali condizioni: in questo caso l'accusato viene espulso in via definitiva;
	\item colpevolezza: l'\EC{} ed il conselor non pongono alcuna condizioni per l'accusato ma lo condanna direttamente all'espulsione;
\end{itemize}

La procedura qui spiegata è quella standard: il conselor può decidere in quelunque momento, senza preavviso, di modificare la procedura per una più oculata gestione del comportamento non consono grave.\todo{l'ho messa perché magari il conselor non c'ha sbatti di stare dietro a noi o perché dice: "ci penso io"}

\subsubsection{non paga}

Dopo una proposta pecuniaria c'è un periodo in cui il treasurer chiede i soldi alla gente.
Sullo scadere di questo periodo (2 settimane), il treasurer notifica tutti i membri attivi che non hanno ancora pagato con una mail in cui si chiede di versare il contributo. Se dopo ulteriori altre 2 settimane un membro attivo non ha ancora versato il proprio contributo allora:

\begin{itemize}
	\item se la spesa non è ancora stata effettuata si rieffettua la votazione pecuniaria da capo aggiornamento la spesa procapite senza considerare come membri attivi tutti i membri che non hanno ancora contribuito. Se la votazione pecuniaria fallisce i soldi precedentemente dati alla casa dai membri vengono redistribuiti.
	\item se la spese è già stata effettuata, i soldi residui vengono prelevati dalla cassa. Se la cassa non copre l'interezza della spesa, si chiede a volontari di ripagare la spesa. Se i volontari non sono sufficienti, la rimanente spese deve essere straordinariamente distribuita su tutti i membri attivi del branch;
\end{itemize}

Il membro attivo che non ha rispettato l'impegno della spesa pecuniaria viene espulso.

\subsubsection{Il membro attivo non espleta i compiti IEEE}

\todo{cosa vuol dire non espleta? perché se lo dice chiaro e tondo ok, se fa lo gnorri mica tanto. E poi, per quanto tempo ci va bene che una persona all'ultimo momento dica:"no, non ce la faccio?"}

Se il membro inadempiente è all'interno dell'\EC{} il chair invia una mail in cui chiede di incontrarsi. All'interno si effettua una riunione con lui per capire perhcé è inadempiente. L'\EC{} (escluso il membro inadempiemte) si raduna ed effetuta una votazione per tenere l'\EC{} inadempiente oppure no. Il membro dell'\EC{} inadempiente (se ritrovato colpevole) viene rimosso dalla carica e votazioni straordinarie sono rilasciate. Se il membro inadempiente non è disponibile all'incontro l'\EC{} decide senza l'incontro.

\todo{scrivere se nell'incontro serve che ci sia tutto l'\EC{} o solo una parte}

Se il membro inadempiente non è all'interno dell'\EC{} il presidente del comitato a cui il membro inadempiente appartiene invia una mail al membro inadempiente in cui chiede di incontrarsi. Nell'incontro si cerca di capire perché il membro è inadempiente. A questo punto il comitato (escluso il membro inadempiente) si riunisce e si determina se il membro è o meno soggetto a ripresa. Se è passibile di richiamo, si informa il membro inadempiente che ha perso una vita.

\todo{scrivere se nell'incontro serve che ci sia tutto il comitato coinvolto o solo una parte}

\subsection{Gestione vite}

Ogni membro ha 3 un massimo di 3 vite. Se un membro perde tutte le vite viene escluso dal STB. Il 1° gennaio di ogni anno ogni membro acquista una vita.

\end{document}