\documentclass[pdf]{article}
\usepackage[utf8]{inputenc}
\usepackage[english]{babel}

\usepackage{amsthm}
\usepackage{xparse}
\usepackage{etoolbox}
\usepackage{xcolor}

\theoremstyle{definition}
\newtheorem{defn}{Definizione}[section]
\newtheorem{regl}{Regola}[section]

%\theoremstyle{remark} %??? Decidere se mantenere questo stile
\newtheorem{prot}{Protocollo}[subsection]

\NewDocumentCommand{\todo}{m}{
	\footnote{\textcolor{red}{#1}}%
}

\NewDocumentCommand{\EC}{}{
	Executive Commitee%
}

\renewcommand{\i}[1]{\textit{#1}}
\renewcommand{\b}[1]{\textbf{#1}}

%opening
\title{Regolamento}
\author{Executive Committee: Lorenzo Nodari, Stefano Prandini, Massimo Bono, Michele Dusi}
\date{\today}

\begin{document}

\maketitle

\newpage

\section{Regolamentazione dei Membri}

\begin{defn}[Iscritta IEEE]
	Una persona viene definita \b{iscritta IEEE} se possiede un account con iscrizione all'IEEE.
\end{defn}

\begin{defn}[Esterno]
	Una persona viene definita \b{esterna} se non è una persona iscritta IEEE.
\end{defn}

\begin{defn}[Status]
	Viene definito \b{status} di una persona iscritta IEEE una qualunque etichetta associabile a tale persona.
\end{defn}

\subsection{Iscrizione allo Student Branch}

\begin{prot}
	\label{Iscrizione allo Student Branch IEEE}
	Il seguente protocollo definisce la procedura per la quale una persona iscritta IEEE guadagna lo status di \i{membro dello Student Branch}.
	\begin{enumerate}
		\item Bla Bla
		\item Bla Bla 2
		\item Bla Bla Bla Bla 3
		\item Terminati i passi precedenti, la persona in oggetto acquisisce immediatamente gli status di \i{membro dello Student Branch} e di \i{membro attivo}.
	\end{enumerate}
\end{prot}

\subsection{Differenziazione fra membri \i{attivi} e \i{passivi}}

%% Da sistemare!

\begin{regl}
	Un iscritto IEEE acquisisce lo status di \b{\i{membro attivo}} dello Student Branch se si verificano contemporaneamente le seguenti condizioni:
	\begin{itemize}
		\item Gode dello status di \i{membro dello Student Branch}.
		\item Manifesta ad altri \i{membri attivi} dello Student Branch la volontà di partecipare alle attività dello Student Branch.
		\item Non ha mai acquisito, in passato, lo status di \i{membro passivo}, sia esso uno status o uno status parametrizzato.
	\end{itemize}
\end{regl}

%% Da sistemare

\begin{regl}
	Un iscritto IEEE guadagna lo status di \b{\i{Membro Passivo}} dello Student Branch se gode dello status di \i{membro attivo} dello Student Branch e si verifica una qualunque delle seguenti condizioni:
	\begin{itemize}
		\item .
		\item 
	\end{itemize}
	Nell'istante in cui un iscritto IEEE guadagna tale status, egli perde immediatamente lo status di \i{membro attivo} dello Student Branch.
\end{regl}

\section{Regolamentazione delle Proposte}

% Definizione delle tipologie di proposte

\begin{defn}[Proposte]
	Proposte.
\end{defn}

\begin{defn}[Proposte Classiche]
	Proposte Classiche.
\end{defn}

\begin{defn}[Proposte Pecuniarie]
	Vengono definite \textbf{proposte pecuniarie} tutte le proposte che coinvolgono flussi di denaro:
	\begin{itemize}
		\item in uscita dalla cassa dello Student Branch per spese ad uso dei membri dello Student Branch.
		\item in uscita dalla cassa dello Student Branch per spese destinate a terzi.
		\item in uscita dai portafogli dei membri attivi dello Student Branch per spese ad uso degli stessi. % DA SISTEMARE
		\item in uscita dai portafogli dei membri attivi dello Student Branch verso la cassa dello Student Branch.
	\end{itemize}
\end{defn}

\begin{defn}[Proposte Straordinarie]
	Vengono definite \textbf{proposte straordinarie} le proposte
\end{defn}

% Differenze fra le due tipologie

% Modalità di presa delle decisioni.
\section{Regolamentazione delle Decisioni}

\subsection{Chi prende le decisioni}

\subsection{Modalità di votazione}

\section{Regolamentazione delle Sanzioni}

Struttura (da valutare):
\begin{enumerate}
	\item Elenco delle "colpe"
	\item Elenco delle relative "sanzioni"
\end{enumerate}

\section{Cose veloci}

\subsection{iscrizione allo student branch}

\begin{enumerate}
	\item lo studente (d'ora in poi denominato con $p$) si iscrive alla IEEE;
	\item verificare che $p$ sia effettivamente iscritta all’IEEE;
	\item il segretario deve richiedere a $p$ la mail relativa al suo account IEEE ed un numero telefonico contattabile;
	\item il segretario iscrive il nuovo studente al google drive, alla mailing list, al sito, open project;
	\item $p$ deve decidere di che comitato far parte;
	\item da quel momento in poi $p$ è un membro attivo dello student branch;
\end{enumerate}

\subsection{Proposte e decisioni}

Si possono fare 3 proposte: classiche, pecuniarie e straordinarie.

le pecuniarie sono quelle proposte che comportano una spesa per ciascun membro attivo dello STB.

le classiche sono quelle proposte che non sono pecuniarie.

Le straordinarie sono proposte classiche che però devono essere decise velocemente (non possono rispettare i tempi definiti dalla procedura per la gestione delle propsote classiche).

\subsubsection{Proposte classiche}

il tutto inizia con una mail inviata al chair da parte di un qualunque membro attivo del STB. Nella mail deve essere contenuta la proposta da decidere: più la proposta è ben scritta maggiori saranno le possibilità che venga accettata. 

La mail viene inviata al chair che a prescindere la inoltra a tutti gli altri membri dell'\EC{}. I membri dell'\EC{} hanno 3 giorni di tempo per visionare la proposta ed esprimere, tramite voto, se proporre la suddetta al resto del branch o meno.
La votazione viene eseguita come segue:
è a maggioranza. Se un membro dell'\EC{} non esprime preferenza non conta. Quindi la proposta per essere accettata deve ottenere:
3/4
2/3
1/2
1/1

Nel caso in cui solo 2 membri dell'\EC{} abbiamo effettuato il voto ed essi siano in disaccordo (caso 1/2) di default la proposta viene rifiutata.

In caso di proposta rifiutata il chair deve inviare una mail al proponente in cui dice:
\begin{itemize}
	\item propsta rifiutata;
	\item motivi (proposta troppo vaga oppure inutile);
\end{itemize}

In caso di proposta accettata il chain deve inviare una mail al proponente in cui dice:
\begin{itemize}
	\item proposta accettata;
\end{itemize}

Il vice deve fare il form da proporre a tutti i membri attivi del branch, nel quale le uniche possibilità di risposta saranno SI'/NO. Ci deve essere una descrizione conforme alla proposta inviata dal proponente. 
Il vice deve fare una mail in cui avverte i membri attivi del branch che è attiva una votazione.

I membri attivi hanno una settimana di tempo per votare. Alla fine della stessa accade che:

\begin{itemize}
	\item la proposta viene accettata dal branch se e solo se raggiunge il 51\% dei votanti ed il 50\%\todo{decidere la percentuale meglio!} dei membri attivi del branch ha votato.
	\item la proposta viene rifiutata altrimenti;
\end{itemize}

In entrambi i casi viene inviata una mail di conferma per  avvisare l'esito della votazione a tutti i membri attivi del branch. La mail è inviata dal vice.

\subsubsection{Proposta pecuniaria}

il tutto inizia con una mail inviata al chair da parte di un qualunque membro attivo del STB. Nella mail deve essere contenuta la proposta da decidere: più la proposta è ben scritta maggiori saranno le possibilità che venga accettata. 

La mail viene inviata al chair che a prescindere la inoltra a tutti gli altri membri dell'\EC{}. I membri dell'\EC{} hanno 3 giorni di tempo per visionare la proposta ed esprimere, tramite voto, se proporre la suddetta al resto del branch o meno.
La votazione viene eseguita come segue:
è a maggioranza. Se un membro dell'\EC{} non esprime preferenza non conta. Quindi la proposta per essere accettata deve ottenere:
3/4
2/3
1/2
1/1

Nel caso in cui solo 2 membri dell'\EC{} abbiamo effettuato il voto ed essi siano in disaccordo (caso 1/2) di default la proposta viene rifiutata.

In caso di proposta rifiutata il chair deve inviare una mail al proponente in cui dice:
\begin{itemize}
	\item propsta rifiutata;
	\item motivi (proposta troppo vaga oppure inutile);
\end{itemize}

In caso di proposta accettata il chain deve inviare una mail al proponente in cui dice:
\begin{itemize}
	\item proposta accettata;
\end{itemize}

Il vice deve fare il form da proporre a tutti i membri attivi del branch, nel quale le uniche possibilità di risposta saranno SI'/NO. Ci deve essere una descrizione conforme alla proposta inviata dal proponente. 
Il vice deve fare una mail in cui avverte i membri attivi del branch che è attiva una votazione. Nella mail il tesoriere deve essere messo in CC.

I membri attivi hanno una settimana di tempo per votare. Alla fine della stessa accade che:

\begin{itemize}
	\item la proposta viene accettata dal branch se e solo se raggiunge il 65\% dei votanti ed il 50\%\todo{decidere la percentuale meglio!} dei membri attivi del branch ha votato.
	\item la proposta viene rifiutata altrimenti;
\end{itemize}

In entrambi i casi viene inviata una mail di conferma per  avvisare l'esito della votazione a tutti i membri attivi del branch. La mail è inviata dal vice.

Il tesoriere ha 2 settimane di tempo per ritirare i soldi dai membri attivi del branch.

\subsubsection{Proposta straordinaria}

Le proposte straordinaria è una proposta classica che però non può ottemperare ai tempi richiesti da una proposta classica.

il tutto inizia con un evento straordinario che richiede una decisione tempestiva.

Prima di tutto l'\EC{} si riunisce ed effettua una votazione peer decidere la proposta.
La votazione è eseguita come segue:
3/4
2/3
1/2 (di defeault no);
1/1

Nel caso in cui solo 2 membri dell'\EC{} abbiamo effettuato il voto ed essi siano in disaccordo (caso 1/2) di default la proposta viene rifiutata.

L'\EC{} deve inviare una mail a tutti i membri attivi del branch in cui spiega la situazione straordinaria e indice la votazione tramite form: 
Il vice deve fare una mail in cui avverte i membri attivi del branch che è attiva una votazione. Ci deve essere una descrizione conforme alla proposta inviata dal proponente. 
L'\EC{} fa un form comunque che se viene compilato da più del 50\% dei membri attivi e ha viene preso in considerazione altrimenti la decisione presa dall'\EC{} viene presa come valida.

I membri attivi hanno il tempo richiesto dalla situazione per votare (esempio: il rettore entra all'incontro del branch e dice: "tra 2 ore dovete decidere". Il tempo di votazione qua sarà di 2 ore).

Alla fine del tempo limite:

\begin{itemize}
	\item la proposta viene accettata dal branch se e solo se raggiunge il 51\% dei votanti ed il 50\%\todo{decidere la percentuale meglio!} dei membri attivi del branch ha votato pro;
	\item la proposta viene rifiutata dal branch se e raggiunge il 51\% dei votanti ed il 50\% dei mambri artivi ha votato contro;
	\item la proposta viene decisa solo dal \EC{} altrimenti;
\end{itemize}

In tutti i casi viene inviata una mail di conferma per  avvisare l'esito della votazione a tutti i membri attivi del branch. La mail è inviata dal vice.

\subsection{sanzioni}

è sanzionabile:

\begin{enumerate}
	\item mancata partecipazione all'attività del branch (un membro attivo non si fa sentire da mesi);
	\item comportamento non consono (un membro attivo rompe una sedia/ruba un computer/stalkera);42
	\item non paga (un membro attivo va contro la decisione presa dal branch di comprare qualcosa e non da i soldi);
	\item non mantiene il lavoro promesso (un membro attivo non collabora nel progetto che lui stesso ha deciso di fare);
	\item non svolge i compiti richiesti dal branch (un membro di un comitato, anche execute, non espleta correttamente i suoi compiti);
\end{enumerate}

\end{document}
